\section{Introduction}

Here is how to include dummy text.
\lipsum[1]
\ExampleComment{Here is an example comment.}
\ExampleCommentInline{Here is an example inline comment.}

Here we define color-blind friendly colors:
\textcolor{kSky}{Sky.}
\textcolor{kBlue}{Blue.}
\textcolor{kGreen}{Green.}
\textcolor{kRed}{Red.}
\textcolor{kPurple}{Purple.}
\textcolor{kYellow}{Yellow.}
\textcolor{kOrange}{Orange.}
\textcolor{kGray}{Gray.}
\textcolor{kLightGray}{LightGray.}
\textcolor{kDarkGray}{DarkGray.}

Here is a problem box.

\probbox{Problem Name}{Description.}{Description.}

%-------------------------------------------------------------------------------
%    1
%-------------------------------------------------------------------------------
\subsection{Question 1}

\begin{enumerate}[label=(\alph*)]
  \item Consider the following algorithm.

  \begin{minipage}{\linewidth}
    \begin{algorithm}[H]
      \SetAlgoLined
      \KwData{this text}
      \KwResult{how to write algorithm with \LaTeX2e }
      initialization\;
      \While{not at end of this document}{
      read current\;
      \eIf{understand}{
      go to next section\;
      current section becomes this one\;
      }{
      go back to the beginning of current section\;
      }
      }
      \caption{How to write algorithms}
    \end{algorithm}
  \end{minipage}

Explain the algorithm.

  \item Running time analysis.
\end{enumerate}

%-------------------------------------------------------------------------------
%    2
%-------------------------------------------------------------------------------
\subsection{Question 2}

\begin{enumerate}[label=(\alph*)]
  \item Your answer.
  \item Consider the following algorithm.

  \begin{minipage}{\linewidth}
    \begin{algorithm}[H]
      \SetKwData{Left}{left}
      \SetKwData{This}{this}
      \SetKwData{Up}{up}

      \SetKwFunction{Union}{Union}
      \SetKwFunction{FindCompress}{FindCompress}

      \KwInput{A bitmap $Im$ of size $w\times l$}
      \KwOutput{A partition of the bitmap}
      \BlankLine
      \emph{special treatment of the first line}\;
      \For{$i \gets 2$ \KwTo $l$}{
      \emph{special treatment of the first element of line $i$}\;
      \For{$j \gets 2$ \KwTo $w$}{\label{forins}
      \Left $\gets$ \FindCompress{$Im[i,j-1]$}\;
      \Up $\gets$ \FindCompress{$Im[i-1,]$}\;
      \This $\gets$ \FindCompress{$Im[i,j]$}\;
      \If(\tcp*[h]{O(\Left,\This)==1}){\Left compatible with \This}{\label{lt}
      \lIf{\Left $<$ \This}{\Union{\Left,\This}}
      \lElse{\Union{\This, \Left}}
      }
      \If(\tcp*[f]{O(\Up,\This)==1}){\Up compatible with \This}{\label{ut}
      \lIf{\Up $<$ \This}{\Union{\Up, \This}}
      \tcp{\This is put under \Up to keep tree as flat as possible}\label{cmt}
      \lElse{\Union{\This, \Up}}\tcp*[h]{\This linked to \Up}\label{lelse}
      }
      }
      \lForEach{element $e$ of the line $i$}{\FindCompress{p}}
      }
      \caption{disjoint decomposition}\label{algo_disjdecomp}
    \end{algorithm}
  \end{minipage}
    
  Running time analysis.

  \item Your answer.
\end{enumerate}
